
\section{Introduction}
\label{sec:introcution}



Lalalalaaa general introduction whz cQED is awesome for simulation and comp. pleeease write me! Lorem ipsum dolor sit amet, consetetur sadipscing elitr, sed diam nonumy eirmod tempor invidunt ut labore et dolore magna aliquyam erat, sed diam voluptua. At vero eos et accusam et justo duo dolores et ea rebum. Stet clita kasd gubergren, no sea takimata sanctus est Lorem ipsum dolor sit amet. Lorem ipsum dolor sit amet, consetetur sadipscing elitr, sed diam nonumy eirmod tempor invidunt ut labore et dolore magna aliquyam erat, sed diam voluptua. At vero eos et accusam et justo duo dolores et ea rebum. Stet clita kasd gubergren, no sea takimata sanctus est Lorem ipsum dolor sit amet. Lorem ipsum dolor sit amet, consetetur sadipscing elitr, sed diam nonumy eirmod tempor invidunt ut labore et dolore magna aliquyam erat, sed diam voluptua. At vero eos et accusam et justo duo dolores et ea rebum. Stet clita kasd gubergren, no sea takimata sanctus est Lorem ipsum dolor sit amet. Lorem ipsum dolor sit amet, consetetur sadipscing elitr, sed diam nonumy eirmod tempor invidunt ut labore et dolore magna aliquyam erat, sed diam voluptua. At vero eos et accusam et justo duo dolores et ea rebum. Stet clita kasd gubergren, no sea takimata sanctus est Lorem ipsum dolor sit amet. 


%To keep a clear structure I focus on the key results, and refer to the literature for derivations and in-depth analysis.


\section{Concepts}
\label{sec:concepts}

In this introductory chapter I present the key concepts, that the reader might find insightful while reading this thesis. I start with a recapitulation of cavity quantum electrodynamics (\textbf{cavity QED}) in section \ref{subsec:introduction:cavity_QED}. The central idea of this project is already developed here: determination of the quantum state of the qubit by exploiting the strong dispersive coupling between qubit and cavity. In section \ref{subsec:introduction:transmon_in_a_3D_cavity} I discuss, in what manner an artificial atom inside a 3D microwave cavity resembles such a cavity QED system. To gain an intuition about this setup, I give a brief introduction on superconducting quantum circuits. I use the simple example of a LC-oscillator, to describe, how an electronic circuit can be treated by means of quantum mechanics (section \ref{subsec:introduction:the_LC_resonator}). The most potent element of quantum circuits is the Josephson-junction (section \ref{subsec:introduction:josephson_junctions}). Its highly non-linear response, whilst being almost dissipationless, allows for the design of remarkable quantum circuits. Most notably, Josephson-junctions can be used to construct artificial atoms - quantum systems whose intrinsic parameters are controllable over a wide range during the fabrication process (section \ref{subsec:introduction:transmon}). Transition frequencies of these artificial quantum systems are typically in the microwave regime.
%TODO: spcidy f range
I introduce a few of the technical terms used in the field of microwave signal processing and describe how weak signals can be described as travelling quantum fields in section \ref{subsec:introduction:microwave_signals}. Finally I present the Josephson-parametric-converter \textbf{JPC} in section \ref{subsec:introduction:jpc}. Using this quantum-limited amplifier as first link in the amplification chain of the measurement setup significantly increases the measurement rate and allows the time-resolved observation of quantum jumps.






\subsection{Cavity QED}
\label{subsec:introduction:cavity_QED}
\cite{haroche_cavity_qed}


The theory of quantum electrodynamics (\textbf{QED}) describes the interaction between matter and light. Matter is comprised by atoms 

The quantized nature of both atoms as well as the electromagnetic field have   

The interaction between matter and light can only be understood in 




\subsubsection{Jaynes-Cummings-model}
\subsubsection{Strong dispersive regime}

\subsection{Quantum Circuits - The LC resonator}
\label{subsec:introduction:the_LC_resonator}


\subsection{Josephson junctions}
\label{subsec:introduction:josephson_junctions}

\subsection{Transmon - the artificial atom}
\label{subsec:introduction:transmon}

\subsubsection{The Cooper-pair-box CPB}
\subsubsection{Transmon regime}


\subsection{A Transmon in a 3D Cavity}
\label{subsec:introduction:transmon_in_a_3D_cavity}



\subsection{Microwave Signals}
\label{subsec:introduction:microwave_signals}

%Fortunately this frequency range lies within the operational band of telecommunication and radar systems and well developed controlling equipment is commercially available. 

\subsubsection{Classical}

\begin{itemize}
\item Transmission line 
\item SNR 
\item Amplifiers: Gain, Noise Temp, Noise figure, ...
\end{itemize}  

\subsubsection{Travelling quantum signals}

\subsection{Josephson-Parametric-Amplifier JPC}
\label{subsec:introduction:jpc}

\subsubsection{Josephson ring modulator}
\subsubsection{Scattering matrix}


