\subsubsection{Fluxmaps}




\begin{figure}
    \centering
    \begin{subfigure}[b]{8.6cm}
        \includegraphics[width=\textwidth]{plots/jpc/fluxmaps/quantumcircuits_signal.pdf}
        \caption{A gull}
        \label{fig:jpc:fluxmaps:reference:signal_resonator}
    \end{subfigure}
    ~ %add desired spacing between images, e. g. ~, \quad, \qquad, \hfill etc. 
      %(or a blank line to force the subfigure onto a new line)
    \begin{subfigure}[b]{8.6cm}
        \includegraphics[width=\textwidth]{plots/jpc/fluxmaps/quantumcircuits_idler.pdf}
        \caption{A tiger}
        \label{fig:jpc:fluxmaps:reference:idler_resonator}
    \end{subfigure}
    \caption{Pictures of animals}\label{fig:jpc:fluxmaps:reference:main}
\end{figure}



\begin{figure}[htb!]
    \centering
        \includegraphics[width=8.6cm]{plots/jpc/fluxmaps/measured_fluxmap_full_range.pdf}
        \caption{Measured Fluxmap of the JPC SN004.}
        \label{fig:jpc:fluxmaps:measured_signal_resonator}
\end{figure}






\FloatBarrier
\subsubsection{Controlling the Gain}




\begin{figure}
\centering
\subcaptionbox{
Phase response of the signal reflected by the JPC signal resonator for three different bias currents while the pump is off.
\label{fig:jpc:tuning:signal_phase_response}}
[8.6cm]{\includegraphics[width=8.6cm]{plots/jpc/tuning/signal_resonator_phase_response.pdf}}
~
\subcaptionbox{
Magnitude of the reflected signal for the same bias current as in (\subref{fig:jpc:tuning:signal_phase_response}) with pump tuned accordingly, to produce a gain. The dips in the data are due to qubits in the waveguide.
\label{fig:jpc:tuning:gain_over_the_whole_range}}
[8.6cm]{\includegraphics[width=8.6cm]{plots/jpc/tuning/gain_curve_sweep.pdf}}
\caption{Controlling the position of the central JPC gain frequency. The signal-resonator is tuned via the bias current until it covers the desired frequency (a). Then the JPC-pump is turned on and its power is carefully increased (a). The gain should be visible in transmission if the frequency matching condition $\omega_\text{S} + \omega_\text{I} = \omega_\text{P}$ is fulfilled. If no gain appears, the pump has to be set to a slightly different frequency with its power again slowly swept from low to high. A decent guess for $\omega_\text{P}$ can be obtained from the provided reference fluxmaps \ref{fig:jpc:fluxmaps:reference:main} for the signal and idler resonator of the JPC.}
\label{fig:jpc:tuning:main}
\end{figure}



\begin{figure}
\centering
{\includegraphics[width=8.6cm]{plots/jpc/tuning/gain_sweep.pdf}}
\caption{Gain curve of the JPC for different pump powers.
}
\label{fig:jpc:tuning:low_gain_to_high_gain}
\end{figure}




