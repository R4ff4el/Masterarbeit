


\subsubsection{From High Power to Low Power}
\label{subsec:characterizing_the_system:from_high_to_low_power}



\begin{figure}[htb!]
\centering
\includegraphics[width=17.2cm]{plots/qubit_spectroscopy/spectra/low_and_high_power.pdf}
\caption{
Reflection spectrum of the cavity taken with a VNA for high (blue) and low power (orange). In the low-power-measurement the average excitation of the cavity filed amounts to roughly $10$ (check this number!) photons. The observed lamb-shift of the resonance frequency is due to the coupling between cavity and transmon.
}
\label{fig:cavity_spectroscopy:from_high_to_power}
\end{figure}







\subsubsection{Driving the Transmon}



\begin{figure}
\centering
\includegraphics[width=17.2cm]{plots/qubit_spectroscopy/multi_plots/fluxmap_qubit_ge_AND_qubit_peak.png}
\caption{testcaption}
\label{testlabel}
\end{figure}




%\begin{figure}
%\centering
%\subcaptionbox{
%Averaged VNA reflection measurements of the cavity (y-axis) while driving the qubit at different frequencies (x-axis).
%\label{fig:cavity_spectroscopy:qubit_ge:contour}}
%[8.6cm]{\includegraphics[width=8.6cm]{plots/qubit_spectroscopy/fluxmaps/driving_ge.pdf}}
%~
%\subcaptionbox{
%Horizontal cut at $f_\text{c}-\chi$ (dashed line). A lorentzian fit of the linearized data yields the qubit frequency $f_\text{01} = \SI{6.449}{GHz}$. 
%\label{fig:cavity_spectroscopy:qubit_ge:horizontal}}
%[8.6cm]{\includegraphics[width=8.6cm]{plots/qubit_spectroscopy/spectra/qubit_ge_transition.pdf}}
%~
%\subcaptionbox{
%A vertical cut at $f_\text{01}$ shows the appearance of the cavity peak corresponding to the excited state of the transmon in the averaged VNA spectrum.
%\label{fig:cavity_spectroscopy:qubit_ge:vertical}}
%[8.6cm]{\includegraphics[width=8.6cm]{plots/qubit_spectroscopy/spectra/two_peaks_low_power.pdf}}
%~
%\subcaptionbox{
%Another VNA spectrum with a stronger qubit driven almost in saturation, as seen by the equal heights of the dips. The dispersive shift is determined by fitting a lorentzian profile two each resonance.
%\label{fig:cavity_spectroscopy:qubit_ge:vertical_more_drive_power}}
%[8.6cm]{\includegraphics[width=8.6cm]{plots/qubit_spectroscopy/spectra/two_peaks_high_power.pdf}}
%\caption{
%Reflection spectrum of the cavity with an additional drive tone to probe the qubit. The qubit drive frequency is iterated in steps of $\SI{100}{kHz}$ from $\SI{6.44}{GHz}$ to $\SI{6.46}{GHz}$ . For each frequency a low power VNA-spectrum with $100$ averages is performed. The reflected magnitude is shown in the contour plot (\subref{fig:cavity_spectroscopy:qubit_ge:contour}). A horizontal cut at $f_\text{c}-\chi$ (dashed line) reveals the qubit resonance at $f_\text{01} = \SI{6.449}{GHz}$ (\subref{fig:cavity_spectroscopy:qubit_ge:horizontal}). Driving the qubit at this frequency populates the first excited level of the transmon. Therefore the cavity resonance corresponding to the excited transmon state becomes more prominent (\subref{fig:cavity_spectroscopy:qubit_ge:vertical}), and even more so for a stronger qubit drive power (\subref{fig:cavity_spectroscopy:qubit_ge:vertical_more_drive_power}).
%}
%\label{fig:cavity_spectroscopy:qubit_ge:main}
%\end{figure}





\begin{figure}
\centering
\includegraphics[width=17.2cm]{plots/qubit_spectroscopy/multi_plots/two_cavity_peaks_different_drives.pdf}
\caption{testcaption}
\label{testlabel}
\end{figure}





\begin{figure}
\centering
\includegraphics[width=17.2cm]{plots/qubit_spectroscopy/multi_plots/fluxmap_qubit_ef_AND_cavity_3_peak_spectrum.png}
\caption{testcaption}
\label{testlabel}
\end{figure}





%
%
%
%\begin{figure}
%\centering
%\subcaptionbox{
%Averaged VNA reflection measurements of the cavity (y-axis) while driving the ge-transition on resonance to populate the excited state of the transmon. Applying an additional RF probe reveals the ef-transition at roughly $\SI{6.177}{GHz}$ (vertical dashed line)
%\label{fig:cavity_spectroscopy:qubit_ef:contour}}
%[8.6cm]{\includegraphics[width=8.6cm]{plots/qubit_spectroscopy/fluxmaps/driving_ef.pdf}}
%~
%\subcaptionbox{
%vertical cut through (\subref{fig:cavity_spectroscopy:qubit_ef:contour}) at $f_\text{c}-\chi$ (dashed line). Driving the ge- and the ef-transition at the same time populates both the $\ket{e}$ and the $\ket{f}$ state of the transmon. The respective peaks appear in the averaged cavity spectrum.
%\label{fig:cavity_spectroscopy:qubit_ef:vertical}}
%[8.6cm]{\includegraphics[width=8.6cm]{plots/qubit_spectroscopy/spectra/three_peaks.pdf}}
%\caption{
%VNA reflection measurement of the cavity with two additional RF tones applied to find the ef-transition of the transmon. Each VNA trace consists of $100$ averages.
%}
%\label{fig:cavity_spectroscopy:qubit_ef:main}
%\end{figure}
%


\begin{figure}
\centering
{\includegraphics[width=8.6cm]{plots/qubit_spectroscopy/spectra/three_peaks_lin.pdf}}
\caption{
Same as above but linear $P_{in}/P_{out}$ data.
}
\label{fig:cavity_spectroscopy:qubit_ef:main}
\end{figure}



